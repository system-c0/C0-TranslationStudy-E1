\documentclass[11pt,a4paper]{article}
\usepackage[margin=2.2cm]{geometry}
\usepackage{fontspec}\usepackage{unicode-math}
\setmainfont{Latin Modern Roman}\setmathfont{Latin Modern Math}
\usepackage{hyperref}\usepackage{parskip}\usepackage{paracol}\usepackage{tcolorbox}

\newcommand{\Eone}[1]{\noindent\textbf{[E1]}~#1}
\newenvironment{disclaimer}{\tcolorbox[colback=black!2,colframe=black!25,
  title=Disclaimer [E1]]}{\endtcolorbox}

\title{System C0: Ontologia Kontrastu\\[4pt]\large Studium Translacji [E1]}
\author{Zespół C0/E1}\date{\today}
\begin{document}\maketitle
\begin{disclaimer}
Ten dokument jest warstwą interpretacyjną \textbf{[E1]}. Nie redefiniuje C0 (brak back‑prop).
\end{disclaimer}
\tableofcontents

\section{Mapa C0 \textrightarrow{} [E1]} % (tabela mapowań)

\section{Case studies}
\subsection{XOR}
\begin{paracol}{2}
\section*{Lewa (C0)}
% W przyszłości wkleimy tu kadry z Package A
\switchcolumn
\section*{Prawa ([E1] readings)}
\Eone{Kategoryczne: …}\\
\Eone{Procesowe: …}\\
\Eone{Ostrzeżenie: to tylko czytania.}
\end{paracol}

\section{Opór języka \& firewall}
\Eone{Praktyki obronne, tagowanie [E1], brak back‑prop.}

\end{document}